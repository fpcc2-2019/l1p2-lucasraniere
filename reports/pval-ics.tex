\documentclass[]{article}
\usepackage{lmodern}
\usepackage{amssymb,amsmath}
\usepackage{ifxetex,ifluatex}
\usepackage{fixltx2e} % provides \textsubscript
\ifnum 0\ifxetex 1\fi\ifluatex 1\fi=0 % if pdftex
  \usepackage[T1]{fontenc}
  \usepackage[utf8]{inputenc}
\else % if luatex or xelatex
  \ifxetex
    \usepackage{mathspec}
  \else
    \usepackage{fontspec}
  \fi
  \defaultfontfeatures{Ligatures=TeX,Scale=MatchLowercase}
\fi
% use upquote if available, for straight quotes in verbatim environments
\IfFileExists{upquote.sty}{\usepackage{upquote}}{}
% use microtype if available
\IfFileExists{microtype.sty}{%
\usepackage{microtype}
\UseMicrotypeSet[protrusion]{basicmath} % disable protrusion for tt fonts
}{}
\usepackage[margin=1in]{geometry}
\usepackage{hyperref}
\hypersetup{unicode=true,
            pdftitle={P-valores x ICs},
            pdfauthor={Lucas Raniere Juvino Santos},
            pdfborder={0 0 0},
            breaklinks=true}
\urlstyle{same}  % don't use monospace font for urls
\usepackage{color}
\usepackage{fancyvrb}
\newcommand{\VerbBar}{|}
\newcommand{\VERB}{\Verb[commandchars=\\\{\}]}
\DefineVerbatimEnvironment{Highlighting}{Verbatim}{commandchars=\\\{\}}
% Add ',fontsize=\small' for more characters per line
\usepackage{framed}
\definecolor{shadecolor}{RGB}{248,248,248}
\newenvironment{Shaded}{\begin{snugshade}}{\end{snugshade}}
\newcommand{\AlertTok}[1]{\textcolor[rgb]{0.94,0.16,0.16}{#1}}
\newcommand{\AnnotationTok}[1]{\textcolor[rgb]{0.56,0.35,0.01}{\textbf{\textit{#1}}}}
\newcommand{\AttributeTok}[1]{\textcolor[rgb]{0.77,0.63,0.00}{#1}}
\newcommand{\BaseNTok}[1]{\textcolor[rgb]{0.00,0.00,0.81}{#1}}
\newcommand{\BuiltInTok}[1]{#1}
\newcommand{\CharTok}[1]{\textcolor[rgb]{0.31,0.60,0.02}{#1}}
\newcommand{\CommentTok}[1]{\textcolor[rgb]{0.56,0.35,0.01}{\textit{#1}}}
\newcommand{\CommentVarTok}[1]{\textcolor[rgb]{0.56,0.35,0.01}{\textbf{\textit{#1}}}}
\newcommand{\ConstantTok}[1]{\textcolor[rgb]{0.00,0.00,0.00}{#1}}
\newcommand{\ControlFlowTok}[1]{\textcolor[rgb]{0.13,0.29,0.53}{\textbf{#1}}}
\newcommand{\DataTypeTok}[1]{\textcolor[rgb]{0.13,0.29,0.53}{#1}}
\newcommand{\DecValTok}[1]{\textcolor[rgb]{0.00,0.00,0.81}{#1}}
\newcommand{\DocumentationTok}[1]{\textcolor[rgb]{0.56,0.35,0.01}{\textbf{\textit{#1}}}}
\newcommand{\ErrorTok}[1]{\textcolor[rgb]{0.64,0.00,0.00}{\textbf{#1}}}
\newcommand{\ExtensionTok}[1]{#1}
\newcommand{\FloatTok}[1]{\textcolor[rgb]{0.00,0.00,0.81}{#1}}
\newcommand{\FunctionTok}[1]{\textcolor[rgb]{0.00,0.00,0.00}{#1}}
\newcommand{\ImportTok}[1]{#1}
\newcommand{\InformationTok}[1]{\textcolor[rgb]{0.56,0.35,0.01}{\textbf{\textit{#1}}}}
\newcommand{\KeywordTok}[1]{\textcolor[rgb]{0.13,0.29,0.53}{\textbf{#1}}}
\newcommand{\NormalTok}[1]{#1}
\newcommand{\OperatorTok}[1]{\textcolor[rgb]{0.81,0.36,0.00}{\textbf{#1}}}
\newcommand{\OtherTok}[1]{\textcolor[rgb]{0.56,0.35,0.01}{#1}}
\newcommand{\PreprocessorTok}[1]{\textcolor[rgb]{0.56,0.35,0.01}{\textit{#1}}}
\newcommand{\RegionMarkerTok}[1]{#1}
\newcommand{\SpecialCharTok}[1]{\textcolor[rgb]{0.00,0.00,0.00}{#1}}
\newcommand{\SpecialStringTok}[1]{\textcolor[rgb]{0.31,0.60,0.02}{#1}}
\newcommand{\StringTok}[1]{\textcolor[rgb]{0.31,0.60,0.02}{#1}}
\newcommand{\VariableTok}[1]{\textcolor[rgb]{0.00,0.00,0.00}{#1}}
\newcommand{\VerbatimStringTok}[1]{\textcolor[rgb]{0.31,0.60,0.02}{#1}}
\newcommand{\WarningTok}[1]{\textcolor[rgb]{0.56,0.35,0.01}{\textbf{\textit{#1}}}}
\usepackage{graphicx,grffile}
\makeatletter
\def\maxwidth{\ifdim\Gin@nat@width>\linewidth\linewidth\else\Gin@nat@width\fi}
\def\maxheight{\ifdim\Gin@nat@height>\textheight\textheight\else\Gin@nat@height\fi}
\makeatother
% Scale images if necessary, so that they will not overflow the page
% margins by default, and it is still possible to overwrite the defaults
% using explicit options in \includegraphics[width, height, ...]{}
\setkeys{Gin}{width=\maxwidth,height=\maxheight,keepaspectratio}
\IfFileExists{parskip.sty}{%
\usepackage{parskip}
}{% else
\setlength{\parindent}{0pt}
\setlength{\parskip}{6pt plus 2pt minus 1pt}
}
\setlength{\emergencystretch}{3em}  % prevent overfull lines
\providecommand{\tightlist}{%
  \setlength{\itemsep}{0pt}\setlength{\parskip}{0pt}}
\setcounter{secnumdepth}{0}
% Redefines (sub)paragraphs to behave more like sections
\ifx\paragraph\undefined\else
\let\oldparagraph\paragraph
\renewcommand{\paragraph}[1]{\oldparagraph{#1}\mbox{}}
\fi
\ifx\subparagraph\undefined\else
\let\oldsubparagraph\subparagraph
\renewcommand{\subparagraph}[1]{\oldsubparagraph{#1}\mbox{}}
\fi

%%% Use protect on footnotes to avoid problems with footnotes in titles
\let\rmarkdownfootnote\footnote%
\def\footnote{\protect\rmarkdownfootnote}

%%% Change title format to be more compact
\usepackage{titling}

% Create subtitle command for use in maketitle
\providecommand{\subtitle}[1]{
  \posttitle{
    \begin{center}\large#1\end{center}
    }
}

\setlength{\droptitle}{-2em}

  \title{P-valores x ICs}
    \pretitle{\vspace{\droptitle}\centering\huge}
  \posttitle{\par}
    \author{Lucas Raniere Juvino Santos}
    \preauthor{\centering\large\emph}
  \postauthor{\par}
    \date{}
    \predate{}\postdate{}
  

\begin{document}
\maketitle

{
\setcounter{tocdepth}{2}
\tableofcontents
}
\hypertarget{o-problema}{%
\subsection{O Problema}\label{o-problema}}

Considerando que os dados da wikimedia que usamos no Laboratório 2, faça
uma inferência sobre como é, na população de todas as sessões do site:

\begin{verbatim}
1. A diferença entre o clickthrough rate dos grupos A e B; e
2. A diferença na proporção buscas com zero resultados nos grupos A e B
\end{verbatim}

\hypertarget{o-que-precisa-ser-feito}{%
\subsection{O que precisa ser feito}\label{o-que-precisa-ser-feito}}

Você deve produzir, para os pontos 1 e 2 acima:

\begin{verbatim}
a. Um parágrafo de resposta contendo os números necessários e explicando a sua resposta usando testes de hipótese via pemutação. O parágrafo deve ser estilo o que você colocaria em um artigo - claro, formal e contendo as estatísticas e termos necessários (p-valor, se foram usadas permutações, qual era a estatística do teste, etc.).
b. Um parágrafo de resposta contendo os números necessários e explicando a sua resposta usando ICs. O parágrafo deve ser estilo o que você colocaria em um artigo - claro, formal e contendo as estatísticas e termos necessários (nível de confiança, limites do IC, etc.).
c. Um parágrafo que comenta se/como os pontos a e b acima concordam, e que compara os dois parágrafos em termos de informação e utilidade para alguém tomando decisões na wikimedia.
\end{verbatim}

\hypertarget{os-dados}{%
\subsection{Os dados}\label{os-dados}}

\begin{verbatim}
## Observations: 136,234
## Variables: 9
## $ session_id              <chr> "0000cbcb67c19c45", "0001382e027b2ea4"...
## $ search_index            <dbl> 1, 1, 1, 1, 2, 3, 4, 5, 6, 1, 1, 2, 1,...
## $ session_length          <dbl> 0, 303, 435, 58, 58, 58, 58, 58, 58, 0...
## $ session_start_timestamp <dbl> 2.016030e+13, 2.016031e+13, 2.016031e+...
## $ session_start_date      <dttm> 2016-03-03 15:20:45, 2016-03-07 08:49...
## $ group                   <chr> "b", "b", "b", "a", "a", "a", "a", "a"...
## $ results                 <dbl> 20, 18, 20, 20, 20, 20, 20, 20, 20, 1,...
## $ num_clicks              <dbl> 0, 1, 1, 0, 0, 0, 0, 0, 0, 0, 0, 1, 0,...
## $ first_click             <dbl> NA, 1, 1, NA, NA, NA, NA, NA, NA, NA, ...
\end{verbatim}

\hypertarget{a-diferenca-entre-o-clickthrough-rate-dos-grupos-a-e-b}{%
\subsection{A diferença entre o clickthrough rate dos grupos A e
B:}\label{a-diferenca-entre-o-clickthrough-rate-dos-grupos-a-e-b}}

Calculando a diferença do clickthrough rate de cada grupo (função
\(\theta\)):

\begin{Shaded}
\begin{Highlighting}[]
\NormalTok{theta_ct <-}\StringTok{ }\ControlFlowTok{function}\NormalTok{(df) \{}

\KeywordTok{return}\NormalTok{(}\KeywordTok{abs}\NormalTok{(}\KeywordTok{diff}\NormalTok{(}\KeywordTok{pull}\NormalTok{(}
\NormalTok{        df }\OperatorTok
\StringTok{          }\KeywordTok{group_by}\NormalTok{(session_id, group) }\OperatorTok
\StringTok{          }\KeywordTok{summarise}\NormalTok{(}\DataTypeTok{c_clicks =} \KeywordTok{any}\NormalTok{(num_clicks }\OperatorTok{>}\StringTok{ }\DecValTok{0}\NormalTok{)) }\OperatorTok
\StringTok{          }\KeywordTok{group_by}\NormalTok{(group) }\OperatorTok
\StringTok{          }\KeywordTok{summarise}\NormalTok{(}\DataTypeTok{clickt_rate =} \KeywordTok{sum}\NormalTok{(c_clicks) }\OperatorTok{/}\StringTok{ }\KeywordTok{n}\NormalTok{())))))}
\NormalTok{\}}
\end{Highlighting}
\end{Shaded}

Calculando a diferença da proporção de buscas com zeros resultados de
cada grupo (\(\theta\)):

\begin{Shaded}
\begin{Highlighting}[]
\NormalTok{buscas }\OperatorTok
\StringTok{  }\KeywordTok{filter}\NormalTok{(}\OperatorTok{!}\KeywordTok{is.na}\NormalTok{(results)) }\OperatorTok
\StringTok{  }\KeywordTok{mutate}\NormalTok{(}\DataTypeTok{t_results =}\NormalTok{ results }\OperatorTok{>}\StringTok{ }\DecValTok{0}\NormalTok{) }\OperatorTok
\StringTok{  }\KeywordTok{group_by}\NormalTok{(group) }\OperatorTok
\StringTok{  }\KeywordTok{summarise}\NormalTok{(}\DataTypeTok{rate =} \DecValTok{1} \OperatorTok{-}\StringTok{ }\NormalTok{(}\KeywordTok{sum}\NormalTok{(t_results) }\OperatorTok{/}\StringTok{ }\KeywordTok{n}\NormalTok{()))}
\end{Highlighting}
\end{Shaded}

\begin{verbatim}
## # A tibble: 2 x 2
##   group  rate
##   <chr> <dbl>
## 1 a     0.184
## 2 b     0.186
\end{verbatim}


\end{document}
